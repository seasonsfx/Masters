%!TEX root = thesis.tex
\chapter{Literature review} \label{ch:lit}

\section{Previous work}
Point cloud cleaning is a interactive segmentation problem. Segmentation is by no means a new problem.

\subsection{Taxonomy}
\begin{itemize}
	\item Interactive/Automated
	\item Information: 2D/2.5D/3D/nD, color, multi-sample, intensity
	\item Resolution: High, Medium, Low
	\item Features: Texture, FPFH, Normals, ...
	\item Target: trees, ground, walls, people, generic
	\item Parameters: Low, Medium, High
\end{itemize}


\subsection{Image segmentation}
	Scans are simply a depth maps so 2D image techniques should be investigated
	\subsection{Image gradients}
	\subsection{Edge detection}
	\subsection{Blob extraction}
	\subsection{Contour extraction}

\subsection{Laser scan segmentation}
	\subsection{Point features}
		\subsubsection{Normals}
			Basic building block for other features
			\begin{itemize}
			\item Ways of performing normal estimation
			\item Neighbour search
				\begin{itemize}
				\item KD Trees
				\item Scan grid
				\end{itemize}
			\item Cost vs Quality
			\item Dealing with noise
			\item Used with machine learning algorithms
			\end{itemize}
				
		\subsubsection{Fast point feature histograms (FPFH)}
			\begin{itemize}
			\item Good discrimination for primitive geometric objects
			\item Costly to calculate
			\end{itemize}
			
		\subsubsection{Principle components}
			\begin{itemize}					
			\item Good for finding edges and planes
			\item Solves part of the inverse problem
			\end{itemize}
			

	\subsection{Region growing}
		\begin{itemize}
		\item Grow neighbourhood from seed point
		\item Add neighbour if it satisfies some similarity criteria
		\item Point feature can be used to determine similarity
		\end{itemize}

	\subsection{K-means clustering}
		Problems with non uniform density

	\subsection{Graph cuts}
		\begin{itemize}
		\item Binary classification
		\item Encode point similarity edges
		\item Shown to work well with arbitrary objects
		\item Results are parameter depended
		\end{itemize}

	\subsection{Machine learning}
		\begin{itemize}
		\item Used in navigation \& aerial scans
		\item Support vector machines
		\item Markov models
		\item Conditional random fields
		\item Requires training
		\end{itemize}
		
	% \subsection{Evaluation of literature}
	% 	\begin{itemize}
	% 	\item FPFH seems to discriminate well for primitive geometry. Might be a useful feature to use in tree classification.
	% 	\item Machine learning has been used to classify vegetation. It however requires training and classified datasets are rare. Possibility to train a classifier on the on scans in the application. Unpredictable accuracy.
	% 	\item Graph cuts have been shown be effective for automated object classification. Augmented with more edge information accuracy may be improved with minimal user interaction.
	% 	\end{itemize}

\section{Evaluation techniques}
\begin{itemize}
	\item What was used in previous work and would it be suitable in this instance?
\end{itemize}
	
% \section{User interaction}
% 	\begin{itemize}
% 		\item Level of interactivity expected?
% 		\item Characteristics of interfaces?
% 		\item Basic expected functionality?
% 		\item Suitable evaluation techniques
% 	\end{itemize}